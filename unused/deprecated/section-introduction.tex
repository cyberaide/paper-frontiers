\section{Introduction}

In this paper, we summarize some of the insights that e we obtained for
a High-Performance Computing Big Data Systems applied to the
MLCommons Earthquake Benchmarks Efforts. This includes
insights into the usability of the HPC Big Data Systems, The usage of
an MLCommons benchmarks and the insights from the applicability in
educational efforts.

Benchmarking is an important effort in using HPC Big Data systems.
While using benchmarks we can compare the performance of various
systems.  We can also evaluate the overall performance of the system
and identify potential areas of improvements and optimizations
either on the system side or the algorithmic performance.

While for traditional HPC systems often the pure
computational power is measured such as projected by the Top500, it is
also important to incorporate the file system performance as it can
considerably impact the computation time. This is especially the case
when fast GPUs are used that need to be fed with data at an adequate
rate to perform well. If file systems are too slow the expensive
specialized GPUs can not be adequately utilized.

Benchmarks also offer a common way to communicate the results to its
users that includes users from the educational community. Students
have often an easier time reproducing a benchmark and assessing the
impact of various parameters modified as part of it to explore how an
algorithm may behave. This is especially the case in deep learning
where a variety of hyperparameters can be modified.

Such parameters should include not only parameters related to
the algorithm itself, but also to explore different systems parameters
such as those impacting data access performance or even energy
consumption.

The paper is structured as follows. First, we provide an introduction to
MLCommons (Section \ref{sec:mlcommons}).  Next, we provide some
insights about Machine Learning in education as it relates to this
paper, but can be generalized to other efforts
(Section~\ref{sec:edu-ml}). We then specifically analyze which
insights we gained practically from using MLCommons in educational
efforts (Section~\ref{sec:edu-mlcommons-insights}). After this we
focus on the Earthquake Forecasting application, describe it
(Section~\ref{sec:eq}) and identify specifically our insights in the
data managemnet for this application (Section~\ref{sec:eq-data}).

As the application used is time consuming and hits at the policy
limitations of the educational HPC data system a special workflow
system has been designd to ccordinate the many task needed to conduct
a comprehensie analysis (Section~\ref{sec:workflow-main}). This
includes the creation of an enhanced batch queue mechanism that
bypasses the polocy limitations but makes management of the many jobs
simple (Section~\ref{sec:workflow-sbatch}). In addition we developed a
grahical compute coordiantor that enables to visualize the execution
of the jobs in a generalized simple workflow system
(Section~\ref{sec:workflow-cc}).  To showcase the performance
(Section~\ref{sec:perf-main}) of the eartquake forecasting
application, present data for the runtime
(Section~\ref{sec:perf-runtime}) and for the energy
(Section~\ref{sec:perf-energy}).



todo integrate: 

\citep{las-22-arxiv-workflow-cc}
\citep{las-infogram}
\citep{las-workflow,las07-workflow}

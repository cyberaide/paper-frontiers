\TODO{ALL}{I still do not get from this the notation ... looks like I am too tired Gregor

It's difficult to write out.  Essentially, there's two variables when collecting the data.  The {\em start} which is a time marker and {\em count} which is a count of 2 week collection of records.  The way how the code works is it adds both the start and seek together to identify where to begin collecting the data.  After this point is set, the program steps forward in 2 week groupings to collect the number of {\em count} records to include.  So the 2wk+30AVG results in 30 data points.  The starting date is what makes this difficult as theres common language mixed in (1 year back; 3M; ...).  I agree this is complicated, but I'm not sure there's an easier way to represent this without going too deep and distracting the reader. --Rob

I tried to explain a in a bit more detail. Hopefully, this helps. Basically the whole data is split into 2 week groupings. The model steps through all of these groupings in sequence to determine predictions. It uses up to 1 year, $<=26$ data points, before the current 2 week grouping run on the model as training data and up to 4 years, $<=104$ data points, after the current 2 week grouping. So say the model is taking the first 2 weeks of January 2000, it would take $<=26$ 2 week groupings before that point as training data, so all of 1999 data would be 26 data points, and $<=104$ data points after that point as validation data,, so all of 2000-2003 data and first 2 weeks of January 2004 minus the first 2 weeks in January 2000. Then it repeats this whole sequence again for 2 week grouping (week 3 and 4 of January 2000) until all data points are run this way. The notation says how much training data is used but not how much validation data is used. --Tom}
\section{Insights into Development from the Earthquake Code}

The original code was developed by a single researcher with the goal to create a DL method called tvelop to apply spacial timeseries evolution for multiplw applications including eartquake, hydrology and COVID prediction. The code was developed in a large Python Jupyter notebook on Google Collab. The total number of lines of code was \TODO{line number}. The code included all definitions of variables and hyperparemetes in the code itself.


difficult to maintain and understand for others

easy to develop by author, many experimental aspects

all varables defined in code not config file

lots of graphic outputs for interactive development

How many lines of code??


no use of libraries
limited use of functions
if conditions for different science applications


large code is too dificult to maintain in colab

papermill

mlcommons focus on one science application at a time

students can not comprehend code

rewritten code to just focus on earth quake

rewritten code to add selected hyperparameters into a configuration file


setup

for

training

valiadation 

comparing output

not much use of libraries

choices

development of multiple runs based on variation of additional time based internal hyperparameters,
--> long runtime, no changes to evaluation section in code

take these parameters out and place them in a configuration fil
->   multiple runs needed and caomparision has to be separated fromprg, ;lots of changes to the program, program will run shorter,


libraries for mlcommons benchmarking, cloudmesh
portable way to define data locations via config
experiment permutation over hyperparameters.
* repeated experiements
* separate evaluation and comparision of accuray which was not in original code.
* comparision of accuracy across different hyperparameter searches.

\section{MLCommons}
\label{sec:mlcommons}

MLCommons is a non-profit organization, that has the goal to
accelerate machine learning innovation to benefit everyone with the
help of more than 70 members from industry, academia, and government
\cite{www-mlcommons}. Its main focus is to develop standardized
benchmarks for measuring the performance systems using machine
learning while applying them to various applications.  This includes,
but is not limited to, application areas from healthcare, automotive,
image analysis, and natural language processing. MLCommons is
concerned with benchmarking training \cite{mlperf-training} and
validation algorithms to measure progress over time.  Through this,
MLCommons investigates machine learning efforts in the areas of
benchmarking, datasets in support of benchmarking, and best practices
that leverage machine learning.

MLCommons is organized into several working groups that address topics
such as benchmarking related to training, training on HPC resources,
and inference conducted on data centers, edge devices, mobile devices, and
tiny devices. Best practices are explored in the areas of
infrastructure and power.  In addition, MLCommons also operates
working groups in the areas of Algorithms, DataPerf Dynabench,
Medical, Science, Storage.  The science working group is concerned
with improving the science just a static benchmark \cite{las-22-mlcommons-science}.

A list of selected benchmarks for the working groups focussing on
inference, training, and science are shown in Table~\ref{tab:mlcommons-benchmarks}.


\begin{table}[htb]
  \caption{MLCommons Benchmarks}
  \label{tab:mlcommons-benchmarks}
  \bigskip

  \resizebox{\linewidth}{!}{
  {\footnotesize
  \begin{tabular}{lllllp{6cm}}
    Name & Training & Inference & HPC & Science & Area \\
    \hline
    MiniGo          & \YES & & & &  Neural-network based Go AI, using TensorFlow\\
    Mask R-CNN      & \YES & & & & Instance segmentation, developed on top of Faster R-CNN \\
    DLRM & \YES     & \YES & & &   Deep Learning Recommendation Model \\
    BERT & \YES     & \YES &  & &  Natural Language Processing \\
    ResNet-50 v1.5  & \YES & \YES & & &  Image Classification \\
    RetinaNet & \YES & \YES & & &  Object Detection \\
    RNN-T           & \YES & \YES & & &  Speech Recognition \\
    3D U-Net & \YES & \YES & & &  Medical Imaging \\
    OpenCatalyst & & & \YES & &   Chemical reactions analysis \\
    DeepCam & & & \YES & &  Deep Learning Climate Segmentation Benchmark \\
    CosmoFlow \cite{cosmoflow} & & & \YES & & Cosmology and Nongalactic Astrophysics \\
    Earthquake & & & & \YES &  Earthquake forecasting \\
    Uno & & & & \YES & Predicting tumor response to drugs \\
    Cloudmask & & & & \YES &  Cloud masking \\
    StemDL & & & & \YES & Space group
classification of solid-state materials from Scanning Transmission Electron Micro-
scope (STEM) data using Deep Learning \\
  \end{tabular}
  }
  }

\end{table}

Due to the strong affiliation with industry as well as the integration
of National Labs and Academic High-Performance Computing centers
MLCommons provides a well-positioned starting point for academic
participation. Over the last years, we have participated significantly
in its effort and integrated efforts from MLCommons into our
educational activities. Since its inception, we leveraged the
MLCommons activities and obtained a
number of important educational insights that we discuss in this paper.


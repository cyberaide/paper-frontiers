\subsection{MLCommonsDeep-Learning based Course Curriculumn}

We can utilize the MLCommons effort to center a course curiculumn around it. For this to work the course will be focussing on deep learning while using examples from MLCommons benchmarks as well as additional enhancements into other topics that may not be covered.

In contarst to other courses that may only focus on DL technologies this course will have the requirement to utilize a significant computational resources as for example available on many campuses as part of an HPC or a national scale facility such as Access. Alternatively Google Collab can be used.

The curriculum is devided into ??? sections.

\begin{description}

\item[Course overview and Introduction:] here the overview of the course is provided. Goals and expectations are explained. An introduction to deep learning is provided. This includes   history and applications of deep learning. A basic introduction to optimization technologies and neural networks is given. The connection between MLCommons Applications is presented.

\item[Infrastructure and Benchamrarking:] An overview of MLCommons based deep learning applications and benchmarks are discussed. This will include a wide variety reaching from tiny devices to supercumpters and hyperscale clouds. Google Collab will be introduced. Practical topics such as using ssh, batch queues are discussed. Explicit effort is placed on using a code editor such as pyCHarm or VScode. Elementary software infrastructure is discussed while reviewing python concepts for functions, classes, and code packaging with pip. The use of github is introduced. 
  
\item[Convolutional Neural Networks:] A deeper understanding is taught by focussing on convolutional neural networks (CNNs). The example of Mask R-CNN is explained.

\item[Recurrent Neural Networks:] RNNs are taught and applications of RNNs are discussed. The RNN-T application focusing on speach recognition is presented and analysed

\item[Natural Language Processing:] As natural language processing has such a big impact in industry and academia additional lectures in that are are presented. This includes large language models, analysing text, applications of NLP, language translation, sentiment analysis.
Practical examples are introduced while looking at ChatGPT. From MLcommons the applications DLRM, BERT, RNN-T are discussed.
\item[learningLearning

 \end{description}

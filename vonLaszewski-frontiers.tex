%%% Version 3.4 Generated 2022/06/14 %%%
%%% You will need to have the following packages installed: datetime, fmtcount, etoolbox, fcprefix, which are normally inlcuded in WinEdt. %%%
%%% In http://www.ctan.org/ you can find the packages and how to install them, if necessary. %%%
%%%  NB logo1.jpg is required in the path in order to correctly compile front page header %%%

%\documentclass[utf8]{FrontiersinHarvard}

\documentclass[utf8]{FrontiersinVancouver} % for articles in journals

%\documentclass[utf8]{frontiersinFPHY_FAMS} % Vancouver Reference
%Style (Numbered) for articles in the journals "Frontiers in Physics"
%and "Frontiers in Applied Mathematics and Statistics"

%\setcitestyle{square} % for articles in the journals "Frontiers in Physics" and "Frontiers in Applied Mathematics and Statistics" 

\usepackage{url}
\usepackage{lineno}
\usepackage[hidelinks]{hyperref}
\usepackage{microtype}
\usepackage{subcaption}
\usepackage[onehalfspacing]{setspace}
\usepackage{comment}
\usepackage{xcolor}
\usepackage{todonotes}

\newcommand{\TODO}[1]{\todo[inline]{#1}}

\newcommand{\YES}{yes}
\makeatletter\newcommand{\tableofcontents}{\@starttoc{toc}}\makeaother

\linenumbers


\def\keyFont{\fontsize{8}{11}\helveticabold }

\def\firstAuthorLast{von Laszewski {et~al.}} 
\def\Authors{Gregor von Laszewski\,$^{1,*}$,
 J.P. Fleischer\,$^{1}$
Robert Knuuti\,$^{1}$
 and Geoffrey. C. Fox\,$^{1}$}

% Affiliations should be keyed to the author's name with superscript
% numbers and be listed as follows: Laboratory, Institute, Department,
% Organization, City, State abbreviation (USA, Canada, Australia), and
% Country (without detailed address information such as city zip codes
% or street names).

% If one of the authors has a change of address, list the new address
% below the correspondence details using a superscript symbol and use
% the same symbol to indicate the author in the author list.

\def\Address{$^{1}$
Biocomplexity Institute,
University of Virginia,
% Town Center Four,
% 994 Research Park Boulevard,
 Charlottesville, VA, 22911, USA
}

% The Corresponding Author should be marked with an asterisk Provide
% the exact contact address (this time including street name and city
% zip code) and email of the corresponding author

\def\corrAuthor{Gregor von Laszewski, Biocomplexity Institute,
University of Virginia,
Town Center Four,
994 Research Park Boulevard,
 Charlottesville, VA, 22911, USA
}

\def\corrEmail{laszewski@gmail.com}

\newcommand{\TITLE}{Insights in High-Performance Big Data Systems Gained from
  Educational
  MLCommons Earthquake Benchmarks Efforts}


\begin{document}

% outcomment toc when submitting

{\bf \TITLE}

\bigskip

\tableofcontents

\title{\TITLE}


\onecolumn
\firstpage{1}

\author[\firstAuthorLast ]{\Authors} %This field will be automatically populated
\address{} %This field will be automatically populated
\correspondance{} %This field will be automatically populated

\extraAuth{}

% If there are more than 1 corresponding author, comment this line and
%uncomment the next one.  \extraAuth{corresponding Author2
%\\ Laboratory X2, Institute X2, Department X2, Organization X2,
%Street X2, City X2 , State XX2 (only USA, Canada and Australia), Zip
%Code2, X2 Country X2, email2@uni2.edu}


\maketitle


\begin{abstract}

\section{}

\input{section-abstract}


\tiny \keyFont{ \section{Keywords:} deep learning, benchmarking, hyper
  parameter search, hybrid heterogeneous hyperparameter search,
  earthquake forecasting}

% All article types: you may provide up to 8 keywords; at least 5 are mandatory.

\end{abstract}


\section{Introduction}

todo \citep{las-22-arxiv-workflow-cc}

\citep{las-infogram}
\citep{las-workflow,las07-workflow}


% For Original Research Articles \citep{conference}, Clinical Trial
% Articles \citep{article}, and Technology Reports \citep{patent}, the
% introduction should be succinct, with no subheadings \citep{book}. For
% Case Reports the Introduction should include symptoms at presentation
% \citep{chapter}, physical exams and lab results \citep{dataset}.



\input{section-mlcommons}

\input{section-edu-ml}
\input{section-edu-mlcommons}

\input{section-earthquake}

\section{Insights into Development from the Earthquake Code}

The original code was developed by a single researcher with the goal to create a DL method called tvelop to apply spacial timeseries evolution for multiplw applications including eartquake, hydrology and COVID prediction. The code was developed in a large Python Jupyter notebook on Google Collab. The total number of lines of code was \TODO{line number}. The code included all definitions of variables and hyperparemetes in the code itself.


difficult to maintain and understand for others

easy to develop by author, many experimental aspects

all varables defined in code not config file

lots of graphic outputs for interactive development

How many lines of code??


no use of libraries
limited use of functions
if conditions for different science applications


large code is too dificult to maintain in colab

papermill

mlcommons focus on one science application at a time

students can not comprehend code

rewritten code to just focus on earth quake

rewritten code to add selected hyperparameters into a configuration file


setup

for

training

valiadation 

comparing output

not much use of libraries

choices

development of multiple runs based on variation of additional time based internal hyperparameters,
--> long runtime, no changes to evaluation section in code

take these parameters out and place them in a configuration fil
->   multiple runs needed and caomparision has to be separated fromprg, ;lots of changes to the program, program will run shorter,


libraries for mlcommons benchmarking, cloudmesh
portable way to define data locations via config
experiment permutation over hyperparameters.
* repeated experiements
* separate evaluation and comparision of accuray which was not in original code.
* comparision of accuracy across different hyperparameter searches.

\section{Insights into Data Management}

In data management wi are currently concerned with various aspects of
the data set, the data compression and storage, as well as the data
access speed. We discuss insights in each of them in the next Sections.

\subsection{Data Sets}

When dealing with datasets we typically encounter several issues.
These issues are addressed by the MLCommons benchmarks and
datamanagement activities so that they provide ideal candidates for
education without spending an exorberant amount of time on data. Such
issues typically include access to dato wihout privacy restrictions,
data preprocessing that makes the datau suitable for deep learning,
data labeling in case they are part of a well defined mlcommons
benchmark. Other issues include data bias, noisy or missing data, as
well as overfitting while using training data. Typically the mlcommons
benchmarks will be designed to have no such issuess, or they have
minimal issues. However some benchmarks such as the science group
benchmarks whic are concerned with improving the science will have to
potentially address these issues in order o improve the accuracy. This
could include even injecting new date and different preproocessing
methods.


\subsection{Data compression}

An issue that if of utmost imporatance especially for large data sets
is how the data is represented. For example, for the earthquake
benchmark we found that the original dataset was 11GB big. However we
found that the data can be easily compressed by a factor of 100. This
is significant, as for example in this case the entire dataset can be
stored in Github. The compressed xz archive file is only 21 MB and to
download only the archive file using wget takes 0.253s. In case the
dataset and its repository is downloaded with Git we note that the
entire Git repository is 108MB~\citep{mlcommons-earthquake-data}. To
download this compressed dataset only takes 7.723s. Thus it is
prefered to just download the explictly used data using for example
wget. In both cases the data is compressed. To uncompress the data it
will take an additional 1 minute and 2.522seconds. However if we were
to download the datai in uncompressed form it woudl take approximately
3 huours and 51 seconds.

From this simple example it is clear that MLCommons benchmarks can
provide insights into how data is managed and delivered to for example
large scale compute clusters with many nodes, while utilizing
compressin algorithms. We will next discuss insights into
infrastructure management while using filesystems in HPC resources.
While often object stores are discussed to host such large datsets it
is imperative to identify the units of storage in such object stores.
In our case an object store that would host individual data recorsd is
not useful due to the vast number of data points. Therfore the best
way to store this data even in an object store is as a single entry of
compressed overall data.


\subsection{Data Access}

Besides having proper data and being able do download it efficiently
from the location of storage, it is impertive to be able to access it
in such a way that the GPUs used for deep learning are being feed with
enough data without being idle. The performance results were somewhat
surprising and had a devestating efect on the overall execution time
that were twice as fast on the personal computer while using an
RTX3090 in contrast to using the HPC center recommended filesystems
when using an A100. For this reason we have made a simple test and
measure the performance to read access the various file systems. The
results are shown in Table~\ref{tab:file-performance} which include
various file systems at University of Virginias Rivanna HPC but also a
comparision with a personal computer from a student.

Based on this observation it was infasbale to consider running the
earthquake benchmark on the regular configured HPC nodes as they ran
on some resources for almost 24 hours. This is also the limit the
Rivana system allows for one job. Hence we were allowed to use a
special compute node that has additional NVMe storage avalable and
accessible to us. On those nodes (in the Table listed as
\verb|/localsratch|) we wer able to obtain a very suitable performance
for this application while having a 10 times fold increas in access in
contrast to the scratch file system an almost double the perfomance
given to us on the project file system. The /tmp system although being
fast was for our application not suffiintly large and also performes
slower then the \verb|/localscratch| set up for us. In addition we
also made an experiment using a shared memory based hosted filesystem
in the nodes RAM.


What we learn from this experience is that a HPC system must provide
the fast file system locally available on the nodes to serve the GPUs
adequately. The computer should be designed form that start to not
only have the fastest possible GPUs for large data processing, but
also a very fast filesystem that can keep up with the data input
requirements presented by the GPU. Furthermore in case updated GPUs
are purchased it is not sufficient to just take toe previous
generation motherboard and CPU processor and memory, but to update the
hradware components and include a state of the art compute note. This
often prevents the repurpossing of the nodde while addeing just GPUs.

\begin{table}[htb]
  \caption{Filetransfer performance of various file systems on Rivanna}
  \label{tab:file-performance}
  \begin{center}
  {\footnotesize 
  \begin{tabular}{lllllp{4.5cm}}
    Machine & File systems & \multicolumn{2}{l}{Bandwidth Performance} & Speedup & Description \\
    \hline
    Rivanna & \verb|/scratch/$USER  (sbatch)|     & 30.6MiB/s & 32.1MB/s  & 1.0 & shared scratch space when running in batch mode \\
    Rivanna & \verb|/scratch/$USER (interactive)| & 33.2MiB/s & 34.8MB/s  & 1.1 & shared scratch space when running interacively \\
    Rivanna & \verb|/home/$USER|                    & 40.9MiB/s & 42.9MB/s  & 1.3 & users home directory \\
    Rivanna & \verb|/project/$PROJECTID |     & 100MiB/s  & 105MB/s  & 3.3 & project sppecific filesystem \\
    Personal Computer  & \verb|c:| & 187MiB/s  & 196MB/s  & 6.1 &  file system on a personal computer \\
    Rivanna & \verb|/tmp|                         & 271MiB/s  & 285MB/s  & 8.9 & temporary file system on a node \\
    \hline
    Selected Nodes Rivanna & \verb|/localscratch|  &              384MiB/s & 403MB/s  & 12.6 & Special access to NVMe storage of a special node in the cluster\\
    RAM disk Rivanna  & \verb|/dev/shm/*|      &             461MiB/s & 483MB/s  & 15.1 & simulated filesystem in a RAM disk\\
    \hline                                             
    \end{tabular}
  \end{table}
  \end{center}
  }


\section{Insights into DL Workflows}





\subsection{Cloudmesh-sbatch}

We describe cloudmesh-sbatch

\subsection{Cloudmesh-cc}

\begin{itemize}
\item Different graphics cards

\item Different epochs of training

\item Create workflow for cloudmask
\end{itemize}


\subsubsection{Analytics Service Pipelines}

\paragraph{Motivation.}
In many cases, a big data analysis is split up into multiple
subtasks. These subtasks may be reusable in other analytics
pipelines. Hence it is desirable to be able to specify and use them in
a coordinated fashion allowing the reuse of the logic represented by the
analysis. Users must have a clear understanding of what the analysis
is doing and how it can be invoked and integrated.

\paragraph{Access Requirements.}
The analysis must include a clear and easy-to-understand specification
that encourages reuse and provides sufficient details about its
functionality, data dependency, and performance. Analytics services may
have authentication, autorotation, and access controls built in that
enable access by users controlled by the service providers.



\begin{figure}[htb]
\centering\includegraphics[width=0.75\columnwidth]{images/processes-nist.pdf}
\label{fig:service-interaction}
\caption{Service Interaction.}
\end{figure}


\subsubsection{Workflow Compute Coordinator}

High-performance computing (HPC) is for decades a very important tool
for science. Scientific tasks can be leveraging the processing power
of a supercomputer so they can run at previously unobtainable high
speeds or utilize specialized hardware for acceleration that otherwise
are not available to the user. HPC can be used for analytic programs
that leverage machine learning applied to large data sets to, for
example, predict future values or to model current states. For such
high-complexity projects, there are often multiple complex programs
that may be running repeatedly in either competition or cooperation.
This may include resources in the same or different data centers. We
developed a hybrid multi-cloud analytics service framework that was
created to manage heterogeneous and remote workflows, queues, and
jobs.  It can be used through a Python API, the command line, and a
REST service. It is supported on multiple operating systems like
macOS, Linux, and Windows 10 and 11.  The workflow is specified via an
easy-to-define YAML file.  Specifically, we have developed a library
called Cloudmesh Compute Coordinator (cloudmesh-cc) that adds workflow
features to control the execution of jobs on remote compute resources,
while at the same time leveraging capabilities provided by the local
compute environments to directly interface with graphical
visualizations better suited for the desktop. The goal is to provide
numerous workflows that in cooperation enhances the experience of the
analytics tasks. This includes a REST service and command line tools
to interact with it.


\begin{figure}[htb]
\centering\includegraphics[width=0.7\columnwidth]{images/fastapi-service.png}
\caption{Fast API Workflow Service.}
% better resolution
\label{fig:fastapi-cc}
\end{figure}

\begin{figure}[htb]
    \centering
    \includegraphics[width=0.50\columnwidth]{images/cloudmesh-cc-new.pdf}
    \caption{Architecture Workflow Service.}
    \label{fig:cc-2}
\end{figure}

\begin{figure}[htb]
    \centering
    \includegraphics[width=0.70\columnwidth]{images/cloudmesh-sbatch-new.pdf}
    \caption{Workflow Script Batch Generator.}
    \label{fig:cm-sbatch}
\end{figure}

\begin{figure}[htb]
    \centering
    \includegraphics[width=0.70\columnwidth]{images/cc-1.png}
    \caption{Workflow user interface. }
    \label{fig:cc-3}
\end{figure}


We have tested the framework while running various MNIST application
examples, including include Multilayer Perceptron, LSTM (Long
short-term memory), Auto-Encoder, Convolutional, and Recurrent Neural
Networks, Distributed Training, and PyTorch training.  A much lager
application using earthquake prediction has also been used.

Figure \ref{fig:fastapi-cc} shows the REST specification and
\ref{fig:cc-2} shows the architecture.

\input{section-eq-performance}

% \subsubsection{Federated Analytics Service Catalogue}
% \subsubsection{Catalogue Attributes}
% \subsubsection{Federated analytics service Registries}
% \subsubsection{Registry Attributes}

% \subsection{Resource Accessibility}
% \subsubsection{Resource Management}
% \subsubsection{Security}

% three axis graph: system, software, science

\section{Nomenclature}

\subsection{Resource Identification Initiative}

{\bf Organization:} \verb|RRID:SCR_011743|

\section*{Conflict of Interest Statement}

The authors declare that the research was conducted in the absence of
any commercial or financial relationships that could be construed as a
potential conflict of interest.

\section*{Author Contributions}

{\em GvL} is the lead author and main contributor to this paper. He
has modified and augmented the earthquake paper to include the ability
to execute hyperparameters. {\em JPF} is a student that has
contributed to various aspects of the workflow component of the paper
and to a number of executions and evaluations of experiment runs. {\em
  RK} was a student and has helped together with {\em GVL} in the
implementation of cloudmesh-sbatch and the porting of the effort to
the UVA machine.  {\em GCF} is the author of the earthquake code and
facilitates the interactions with the MLCommons Science Working group
as a group leader of that effort.

\section*{Funding}

Work was in part funded by the NSF CyberTraining: CIC: CyberTraining
for Students and Technologies from Generation Z with the award numbers
1829704 and 2200409 and NIST 60NANB21D151T.  The work was also funded
by the Department of Energy under the grant Award
No. DE-SC0023452. The work was conducted at the Biocomplexity
Institute and Initiative at University of Virginia.

\section*{Acknowledgments}

We like to thank Thomas Butler and Jake Kolessar for their
contributions during the capstone project while focusing on executing
initial runs of the code, and experimenting with modifications to the
code including logging. Please note that since this team finished
their work, significant improvements have been made by the authors of
this paper.

\section*{Data Availability Statement}

The code is all in the public domain and available on GitHub at the following locations

\begin{itemize}

\item {\bf cloudmesh-cc} -- Is a code to control workflows to be executed on
  remote computing
  resources. \url{https://github.com/cloudmesh/cloudmesh-cc}

\item {\bf cloudmesh-sbatch} -- Is a code to generate batch scripts for
  hyperparameter studies high-performance computers so they can be
  executed on different supercomputers by multiple
  accounts. \url{https://github.com/cloudmesh/cloudmesh-sbatch}

\item {\bf cloudmesh} -- Cloudmesh is a large collection of repositories for
  accessing cloud and HPC
  resources. \url{https://github.com/orgs/cloudmesh/repositories}

\item {\bf mlcommons eartchquake production code} -- The MLCommons Sceience
  Working group is described at
  \url{https://mlcommons.org/en/groups/research-science/}. This page
  contains the links to the production-level earthquake code.

\item {\bf MLcommons earthquake development code} -- The development version of
  the code is available in this repository. It also contains many of
  the analysis scripts that are not part of the production code
  hosted by MLCommons \url{https://github.com/laszewsk/mlcommons}.

\end{itemize}


% \bibliographystyle{Frontiers-Harvard}

\bibliographystyle{Frontiers-Vancouver} % Many Frontiers journals
% use the numbered referencing system, to find the style and resources
% for the journal you are submitting to:
% https://zendesk.frontiersin.org/hc/en-us/articles/360017860337-Frontiers-Reference-Styles-by-Journal

\bibliography{vonLaszewski-references}


\end{document}

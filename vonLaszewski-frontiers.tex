%%% Version 3.4 Generated 2018/06/15 %%%
%%% You will need to have the following packages installed: datetime, fmtcount, etoolbox, fcprefix, which are normally inlcuded in WinEdt. %%%
%%% In http://www.ctan.org/ you can find the packages and how to install them, if necessary. %%%
%%%  NB logo1.jpg is required in the path in order to correctly compile front page header %%%

\documentclass[utf8]{frontiersSCNS} % for Science, Engineering and Humanities and Social Sciences articles

\usepackage{url,hyperref,lineno,microtype,subcaption}
\usepackage[onehalfspacing]{setspace}
\usepackage{todonotes}
\newcommand{\TODO}[1]{\todo[inline]{#1}}

\linenumbers

\def\keyFont{\fontsize{8}{11}\helveticabold }
\def\firstAuthorLast{von Laszewski {et~al.}} %use et al only if is more than 1 author
\def\Authors{Gregor von Laszewski\,$^{1,*}$,
 J.P. Fleischer\,$^{1}$
Robert Knuuti\,$^{1}$
 and Geoffrey. C. Fox\,$^{1}$}

% Affiliations should be keyed to the author's name with superscript
% numbers and be listed as follows: Laboratory, Institute, Department,
% Organization, City, State abbreviation (USA, Canada, Australia), and
% Country (without detailed address information such as city zip codes
% or street names).

% If one of the authors has a change of address, list the new address
% below the correspondence details using a superscript symbol and use
% the same symbol to indicate the author in the author list.

\def\Address{$^{1}$
Biocomplexity Institute,
University of Virginia,
% Town Center Four,
% 994 Research Park Boulevard,
 Charlottesville, VA, 22911, USA
}
% The Corresponding Author should be marked with an asterisk
% Provide the exact contact address (this time including street name and city zip code) and email of the corresponding author

\def\corrAuthor{Gregor von Laszewski, Biocomplexity Institute,
University of Virginia,
Town Center Four,
994 Research Park Boulevard,
 Charlottesville, VA, 22911, USA
}

\def\corrEmail{laszewski@gmail.com}




\begin{document}
\onecolumn
\firstpage{1}

\title[Running Title]{Insights in High Performance Big Data Systems
  gained from MLCommons Earthquake Benchmarks} 

\author[\firstAuthorLast ]{\Authors} % This field will be automatically populated
\address{} % This field will be automatically populated
\correspondance{} % This field will be automatically populated

\extraAuth{} % no 2nd corresponding author

\maketitle


\begin{abstract}

%%% Leave the Abstract empty if your article does not require one, please see the Summary Table for full details.

  \section{}

  For full guidelines regarding your manuscript please refer to \href{http://www.frontiersin.org/about/AuthorGuidelines}{Author Guidelines}.

As a primary goal, the abstract should render the general significance
and conceptual advance of the work clearly accessible to a broad
readership. References should not be cited in the abstract. Leave the
Abstract empty if your article does not require one, please see
\href{http://www.frontiersin.org/about/AuthorGuidelines#SummaryTable}{Summary
  Table} for details according to article type.


\tiny
\keyFont{ \section{Keywords:} deep learning, benchmarking, hyper parameter search, hybrid heterogeneous hyper parameter serach, earthquake forecasting }

% All article types: you may provide up to 8 keywords; at least 5 are mandatory.
\end{abstract}

\section{Introduction}

For Original Research Articles \citep{conference}, Clinical Trial
Articles \citep{article}, and Technology Reports \citep{patent}, the
introduction should be succinct, with no subheadings \citep{book}. For
Case Reports the Introduction should include symptoms at presentation
\citep{chapter}, physical exams and lab results \citep{dataset}.



\section{Article types}

For requirements for a specific article type please refer to the
Article Types on any Frontiers journal page. Please also refer to
\href{http://home.frontiersin.org/about/author-guidelines#Sections}{Author
  Guidelines} for further information on how to organize your
manuscript in the required sections or their equivalents for your
field

% For Original Research articles, please note that the Material and
% Methods section can be placed in any of the following ways: before
% Results, before Discussion or after Discussion.

\section{Manuscript Formatting}

\subsection{Heading Levels}

%There are 5 heading levels

\subsection{Level 2}
\subsubsection{Level 3}
\paragraph{Level 4}
\subparagraph{Level 5}

\subsection{Equations}
Equations should be inserted in editable format from the equation editor.

\begin{equation}
\sum x+ y =Z\label{eq:01}
\end{equation}

\subsection{Figures}

Frontiers requires figures to be submitted individually, in the same
order as they are referred to in the manuscript. Figures will then be
automatically embedded at the bottom of the submitted
manuscript. Kindly ensure that each table and figure is mentioned in
the text and in numerical order. Figures must be of sufficient
resolution for publication
\href{http://home.frontiersin.org/about/author-guidelines#ResolutionRequirements}{see
  here for examples and minimum requirements}. Figures which are not
according to the guidelines will cause substantial delay during the
production process. Please see
\href{http://home.frontiersin.org/about/author-guidelines#GeneralStyleGuidelinesforFigures}{here}
for full figure guidelines. Cite figures with subfigures as figure
\ref{fig:2}B.


\subsubsection{Permission to Reuse and Copyright}

Figures, tables, and images will be published under a Creative Commons
CC-BY licence and permission must be obtained for use of copyrighted
material from other sources (including
re-published/adapted/modified/partial figures and images from the
internet). It is the responsibility of the authors to acquire the
licenses, to follow any citation instructions requested by third-party
rights holders, and cover any supplementary charges.

%% Figures, tables, and images will be published under a Creative
%% Commons CC-BY licence and permission must be obtained for use of
%% copyrighted material from other sources (including
%% re-published/adapted/modified/partial figures and images from the
%% internet). It is the responsibility of the authors to acquire the
%% licenses, to follow any citation instructions requested by
%% third-party rights holders, and cover any supplementary charges.

\subsection{Tables}

Tables should be inserted at the end of the manuscript. Please build
your table directly in LaTeX.Tables provided as jpeg/tiff files will
not be accepted. Please note that very large tables (covering several
pages) cannot be included in the final PDF for reasons of space. These
tables will be published as
\href{http://home.frontiersin.org/about/author-guidelines#SupplementaryMaterial}{Supplementary
  Material} on the online article page at the time of acceptance. The
author will be notified during the typesetting of the final article if
this is the case.

\section{Nomenclature}

\subsection{Resource Identification Initiative}

{\bf Organization:} RRID:SCR_011743

\section*{Conflict of Interest Statement}

The authors declare that the research was conducted in the absence of
any commercial or financial relationships that could be construed as a
potential conflict of interest.

\section*{Author Contributions}

GvL is the leadauthor and main contributr to this paper. He has
modified and augmented the earthquake paper to include the ability to
execute hyperparameters. 

JPF is a student hthat has contributed to various aspects of the
workflow component of the paper and to a number of executions and
evaluations of experiment runs.

RK has helped togetehr with GVL in teh implementation of the secign of
cloudmesh-sbatch and the porting of the effort to the UVA machine.

GCF is the author of the earthquake code and facilitaties the
interactions with the MLCommons Science Working group as a group
leader of that effort.

\section*{Funding}

Details of all funding sources should be provided, including grant numbers if applicable. Please ensure to add all necessary funding information, as after publication this is no longer possible.

\section*{Acknowledgments}

Work was in part funded by the NSF CyberTraining: CIC:
CyberTraining for Students and Technologies from Generation Z with the
award numbers 1829704 and 2200409 and NIST 60NANB21D151T. 
The work was also funded by the \TODO{Department of Energy under the
  grant ???.}

We like to thank Thomas Butler and Jake Kolessar for their
contributions during the capstone project while focussing on executing
initial runs of the code, and makeing modifications. In particular we
like to thank for the correction of the desnity polts of the
earthquake locations, and Jake for his efforts in alalysing the impact
of the energy based on the script to collect log data provided by GVL. 
The code was since modified with a much updated logging library and an
improved batch management by GVL and RK.

\section*{Supplemental Data}

\href{http://home.frontiersin.org/about/author-guidelines#SupplementaryMaterial}{Supplementary
  Material} should be uploaded separately on submission, if there are
Supplementary Figures, please include the caption in the same file as
the figure. LaTeX Supplementary Material templates can be found in the
Frontiers LaTeX folder.

\section*{Data Availability Statement}

The datasets [GENERATED/ANALYZED] for this study can be found in the
[NAME OF REPOSITORY] [LINK].

% Please see the availability of data guidelines for more information,
% at
% https://www.frontiersin.org/about/author-guidelines#AvailabilityofData

\bibliographystyle{frontiersinSCNS_ENG_HUMS}

\bibliography{test}

%%% Make sure to upload the bib file along with the tex file and PDF
%%% Please see the test.bib file for some examples of references

\section*{Figure captions}

%%% permitted 15 figures and tables, one figure with
%%% multiple subfigures will count as only one figure.

\begin{figure}[h!]
\begin{center}
% \includegraphics[width=10cm]{logo1}% This is a *.eps file
\end{center}
\caption{Enter the caption for your figure here.  Repeat as  necessary for each of your figures}\label{fig:1}
\end{figure}


\begin{figure}[h!]
\begin{center}
% \includegraphics[width=15cm]{logos}
\end{center}
\caption{This is a figure with sub figures, \textbf{(A)} is one logo, \textbf{(B)} is a different logo.}\label{fig:2}
\end{figure}

%%% If you are submitting a figure with subfigures please combine
%%% these into one image file with part labels integrated.

% we do this in separate latex file and create a pdf?

\end{document}
